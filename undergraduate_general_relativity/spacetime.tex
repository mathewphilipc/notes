 \documentclass[12 pt]{article}
\usepackage{amsmath, amssymb, mathtools, slashed, amsfonts}
\usepackage[margin=1in]{geometry}
\usepackage{tikz-cd}
\usepackage{tikz} 
\usepackage{tikz-feynman}
\tikzfeynmanset{compat=1.1.0}
\usepackage[shortlabels]{enumitem}
\usepackage{graphicx}

\setlength{\parindent}{0pt}      % no paragraph indentation
\setlength{\parskip}{1.0em}      % vertical space between paragraphs



% Commonly used sets of numbers
\newcommand{\R}{\mathbb{R}}
\newcommand{\Z}{\mathbb{Z}}
\newcommand{\C}{\mathbb{C}}
\newcommand{\Q}{\mathbb{Q}}
\newcommand{\N}{\mathbb{N}}
\newcommand{\A}{\mathbb{A}}
\newcommand{\PR}{\mathbb{P}}
\newcommand{\F}{\mathbb{F}}


% Shortcuts for text formatting
\newcommand{\tb}[1]{\textbf{#1}}
\newcommand{\ti}[1]{\textit{#1}}

% Math fonts
\newcommand{\mf}[1]{\mathfrak{#1}}
\newcommand{\mc}[1]{\mathcal{#1}}

% Shortcuts for inner product spaces
\newcommand{\KET}[1]{\left| #1 \right\rangle }
\newcommand{\BRA}[1]{\left\langle #1 \right| }
\newcommand{\IP}[2]{\left\langle #1 \left| #2 \right\rangle \right.}
\newcommand{\Ip}[2]{\left\langle #1, #2 \right\rangle}
\newcommand{\nm}[1]{\left\| #1 \right\|}

% Shortcuts for the section of 3D rotations
\newcommand{\lo}{\textbf{L}_1}
\newcommand{\ltw}{\textbf{L}_2}
\newcommand{\lt}{\textbf{L}_3}


% Shortcuts for geometric vectors
\newcommand{\U}{\textbf{u}}
\newcommand{\V}{\textbf{v}}
\newcommand{\W}{\textbf{w}}
\newcommand{\B}[1]{\mathbf{#1}}
\newcommand{\BA}[1]{\hat{\mathbf{#1}}}

% Other shortcuts

\newcommand{\G}{\gamma}
\newcommand{\LA}{\mathcal{L}}


\newcommand{\LP}{\left(}
\newcommand{\RP}{\right)}

\newcommand{\PA}[2]{\frac{\partial #1}{\partial #2}}

\newcommand{\HI}{\mathcal{H}}
\newcommand{\AL}{\mathcal{A}}

\newcommand{\D}{\partial}

\newcommand{\bs}{\textbackslash}

\newcommand{\T}{\mathcal{T}}
\newcommand{\arr}{\mathcal{R}}

\numberwithin{equation}{section}
\setcounter{section}{0}





\def\Xint#1{\mathchoice
{\XXint\displaystyle\textstyle{#1}}%
{\XXint\textstyle\scriptstyle{#1}}%
{\XXint\scriptstyle\scriptscriptstyle{#1}}%
{\XXint\scriptscriptstyle\scriptscriptstyle{#1}}%
\!\int}
\def\XXint#1#2#3{{\setbox0=\hbox{$#1{#2#3}{\int}$ }
\vcenter{\hbox{$#2#3$ }}\kern-.6\wd0}}
\def\ddashint{\Xint=}
\def\dashint{\Xint-}




\begin{document}

\title{Notes on Spacetime}
\author{Mathew Calkins\\
  \textit{Center for Cosmology and Particle Physics},\\
  \textit{New York University}\\
  \texttt{mc8558@nyu.edu}.
}

\date{\today}

\maketitle


\begin{abstract}
Some supplementary discussion of spacetime in special relativity.
\end{abstract}


{\setlength{\parskip}{0pt}\tableofcontents}



\section{Galileo, Poincare, and Lorentz}
During lecture we discussed what are classically taken to be the two axioms of special relativity: \begin{itemize}
\item Any two reference frames defined by inertial observers are equally valid. The laws of physics appear the same to every inertial observer. 
\item In any two inertial reference frames, the vacuum speed of light takes the same value. In SI units this is $c = 2.998 \times 10^8$ m$/$s, while in natural units this is simply $c = 1$. No physical signal can exceed this speed.
\end{itemize}
We also examined the consequences of these axioms via a number of carefully arranged physical thought experiments wherein our friends Alice and Bob interacted with mirrors and trains and flashbulbs and so on. These are important to meditate on, but here we skip to the results. The key consequence of these axioms is that the set of transformations which relate inertial frames to inertial frames has changed.



\subsection{Galilean Transformations}
In Galilean relativity (the kind you learned about when studying Newtonian mechanics), inertial observers are related by three types of transformations. We can describe these neatly by considering an arbitrary spacetime event which one observer measures to happen at $(t,x,y,z)$ while another measures $(t',x',y',z')$. The first class of Galilean transformations are rotations about any spatial axis, such as the following rotation about the $z$ axis \begin{equation*}
\begin{bmatrix}
x' \\ y'
\end{bmatrix} = \begin{bmatrix}
\cos \theta & - \sin \theta \\
\sin \theta & \cos \theta
\end{bmatrix} \begin{bmatrix}
x \\ y
\end{bmatrix}
\end{equation*}
with $t = t'$ and $z = z'$ unchanged. The second are \ti{Galilean boosts} along any spatial direction, and spacetime translations. A Galilean boost along the $x$ direction by a velocity $v$ is given by \begin{equation*}
x' = x - vt
\end{equation*}
with all other coordinates untouched. For comparison with Lorentz boosts it will be illustrative to write this too in matrix form as \begin{equation*}
\begin{bmatrix}
t' \\ x'
\end{bmatrix} = \begin{bmatrix}
1 & 0 \\
-v & 1
\end{bmatrix} \begin{bmatrix}
t \\ x
\end{bmatrix}.
\end{equation*}
Finally, we have a set of transformations so simple that they are sometimes left unsaid, namely the \ti{spacetime translations}. These are transformations of the form \begin{align*}
t' & = t - a_t, \\
\vec{x}' & = \vec{x} - \vec{a}
\end{align*}
for $a_t,\vec{a}$ constants. These correspond to shifting the origin of our spatial coordinate system by the constant vector $\vec{a}$ or changing when we start our stopwatch by an amount $a_t$. The \ti{Galilean group} is the set of all of these transformations, plus anything new we can build up by composing these together.

\tb{Exercise} Convince yourself that, if we allow arbitrary compositions, the resulting group is exactly all the transformations of the form \begin{align*}
t' & = t - a_t, \\
\vec{x}' & = R \vec{x} - \vec{v} t - \vec{a}
\end{align*}
for $a_t,\vec{a}, \vec{v}$ constants and $R$ a rotation matrix (abstractly a special orthogonal matrix, concretely one satisfying $R^T = R^{-1}$ and $\det(R) = 1$).

As we noted above, the pure translation part is comparatively boring, so we often don't bother studying it. We can do this arbitrarily by simply declaring that we only consider the Galilean transformations with $\vec{a},a_t$ vanishing. Slightly less arbitrarily, we can ignore these by studying only the \tb{differences} between events. Given two different events with coordinates $(t_1,\vec{x}_1)$ and $(t_2, \vec{x}_2)$ we can consider the coordinates of the second event relative to the first \begin{align*}
\Delta \vec{x} & := \vec{x}_2 - \vec{x}_1, \\
\Delta t & := t_2 - t_1
\end{align*}
plus the corresponding statement in the primed coordinate system. Then from the individual transformation laws \begin{align*}
t_i' & = t_i - a_t, \\
\vec{x}_i' & = R \vec{x}_i - \vec{v} t_i - \vec{a}
\end{align*}
we see that $\vec{a}, a_t$ drop out when we subtract, so the differences obey the simpler laws \begin{align*}
\Delta t' & = \Delta t, \\
\Delta \vec{x}' & = R \Delta \vec{x} - \vec{v} \Delta t.
\end{align*}
%We can see from this point how Galilean relativity treats time and space very differently. The requirement that $R$ be a rotation matrix is put in place so that we can only mix up our spatial coordinates in a way that respects the 


\subsection{Lorentz and Poincar\'{e}}
Now we can say in a very compact way how the transformation rules of special relativity differ from those known to Newton. Just as before we have a basic set of transformations - rotations, boosts, and spacetime translations - which are guaranteed to take you from one inertial frame to another. Translations and rotations work exactly as before, but boosts work slightly differently. If we boost by a velocity $v$ along the $x$ direction, instead of the Galilean transformation rule \begin{equation*}
\begin{bmatrix}
t' \\ x'
\end{bmatrix} = \begin{bmatrix}
1 & 0 \\
-v & 1
\end{bmatrix} \begin{bmatrix}
t \\ x
\end{bmatrix}.
\end{equation*}
we now have the \ti{Lorentz boost} \begin{equation*}
\begin{bmatrix}
t' \\ x'
\end{bmatrix} = \gamma_v \begin{bmatrix}
1 & -v \\
-v & 1
\end{bmatrix} \begin{bmatrix}
t \\ x
\end{bmatrix} \mbox{ where } \gamma_v := \frac{1}{\sqrt{1 - v^2}}.
\end{equation*}
As a sanity check we should verify that for boosts by velocities much smaller than $c$, this approximately reduces to our Galilean rule. A convenient way to do this is to explicitly restore all factors of $c$, and then work in the limit where $c \gg v$\footnote{More precisely, we take the limit $v/c \to 0$ while carefully not taking $v \to 0$}. We can do this in a unique way (up to common global constants between the two sides of our equation) just by insisting that all of the dimensions make sense in the SI system of units. So we replace each $t \to ct$ so that our coordinates have the same units in all of their components, then replace each $v \to v/c$ so that $\gamma$ and the transformation matrix are both dimensionless. The result is \begin{equation*}
\begin{bmatrix}
ct' \\ x'
\end{bmatrix} = \gamma_v \begin{bmatrix}
1 & -v/c \\
-v/c & 1
\end{bmatrix} \begin{bmatrix}
ct \\ x
\end{bmatrix} \mbox{ where } \gamma_v := \frac{1}{\sqrt{1 - v^2/c^2}}
\end{equation*}
or after moving terms around to make our analysis clearer \begin{equation*}
\begin{bmatrix}
t' \\ x'
\end{bmatrix} = \gamma_v \begin{bmatrix}
1 & -v/c^2 \\
-v & 1
\end{bmatrix} \begin{bmatrix}
t \\ x
\end{bmatrix}.
\end{equation*}
So in the limit where $v$ is much smaller than $c$ this becomes \begin{equation*}
\begin{bmatrix}
t' \\ x'
\end{bmatrix} \approx \begin{bmatrix}
1 & 0 \\
-v & 1
\end{bmatrix} \begin{bmatrix}
t \\ x
\end{bmatrix} \mbox{ since } \gamma_v \approx 1.
\end{equation*}
As we claimed, Lorentz boosts by velocities much smaller than $c$ are approximately Galilean boosts. This kind of analysis is broadly useful for predicting relativistic corrections to Galilean physics. We can take the expression above for the boost (or any other quantity you like from relativistic physics), and instead of going all the way to $v/c \to 0$ we expand as a Taylor series in powers $(v/c)^n$ about $v/c = 0$. You may have seen this kind of analysis used to study the hyperfine splitting in the hydrogen atom.

The group of transformations that can be built up from rotations, Lorentz boosts, and spacetime translations is called the \ti{Poincar\'{e} group}. If we don't include spacetime translations (which we often don't, if only because they aren't very interesting) we get the \ti{Lorentz group}, whose elements are called \ti{Lorentz transformations}.


\section{Timelike, Spacelike, Lightlike}

\subsection{Spacetime Intervals and the Speed of Light}
We noted in lecture that there is a close analogy between Lorentz transformations and rotations. Just as rotations about the origin respect the distance between any point and the origin \begin{equation*}
{x'}^2 + {y'}^2 + {z'}^2 = x^2 + y^2 + z^2
\end{equation*}
Lorentz transformations respect the slightly different quadratic form \begin{equation*}
- c^2 {t'}^2 + {x'}^2 + {y'}^2 + {z'}^2 = - c^2 t^2 + x^2 + y^2 + z^2
\end{equation*}
where we have restored explicit factors of $c$ for the moment. If we include spacetime translation (that is, we consider the full Poincar\'{e} group, not just the Lorentz group) then we get the slightly weaker statement that our transformations preserve the \ti{invariant interval} between events \begin{equation*}
- c^2 (\Delta t')^2 + (\Delta x')^2 + (\Delta y')^2 + (\Delta z')^2 = - c^2 (\Delta t)^2 + (\Delta x)^2 + (\Delta y)^2 + (\Delta z)^2
\end{equation*}
From this we can neatly confirm that our mathematical framework does indeed reproduce one of our initial physical axioms. Suppose that, in our unprimed frame, the two events with coordinate difference $\LP \Delta t, \Delta \vec{x} \RP$ in the unprimed frame lie along the trajectory of a photon, which moves along a straight line at the speed of light. Directly from the definition of speed, we see that the spatial and temporal parts of the difference between the two events must be related as \begin{equation*}
\frac{\nm{\Delta \vec{x}}}{|\Delta t|} = \mbox{speed of photon} = c.
\end{equation*}
But this is exactly the same as the statement that the invariant interval vanishes. \begin{equation*}
- c^2 (\Delta t')^2 + (\Delta x')^2 + (\Delta y')^2 + (\Delta z')^2 = 0.
\end{equation*}
So the statement \begin{center}
\ti{If the first inertial observer sees the invariant interval between two events vanish, so will the second.}
\end{center}
in terms of an object that passes through both events along a straight line becomes \begin{center}
\ti{If the first inertial observer sees an object in uniform motion moving at speed $c$, the second does too.}
\end{center}
which was one of our starting postulates: everyone must agree on the speed of light!

Inspired by the physical set up involving a photon, we sometimes say that these two events are \ti{lightlike-separated} or that the displacement vector $(\Delta t, \Delta \vec{x})$ is a \ti{lightlike} vector. Inspired by the equivalent story involving the vanishing of the invariant interval, we sometimes instead say that the events are \ti{null-separated} or that their displacement vector is a \ti{null} vector.

Given some fixed event with coordinates $(t_0, \vec{x}_0)$, the set up of all points which are lightlike-separated from $(t_0, \vec{x}_0)$ is called its \ti{lightcone}, as inspired by the following diagram.\begin{figure}[ht]
\centering
\includegraphics[width=0.5\textwidth]{lightcone.png}
\caption{A lightcone in a (2+1)-dimensional spacetime.}
\label{Fig:lightcone}
\end{figure}
We can say a great deal about causality in terms of this diagram. Since the speed of light is the upper bound for all physical signal transmission, the interior of the future-directed part of the lightcone is exactly the set of all events that $(t_0, \vec{x}_0)$ can influence. Similarly, the interior of the past-directed lightcone is the set of all events that can influence the physics at $(t_0, \vec{x}_0)$.




\subsection{Timelike- and Spacelike-Separated Events}
We have seen that events which are lightlike-separated stay lightlike-separated under Lorentz transformations. What about other events? If two events are separated in such a way that their invariant interval is negative \begin{equation*}
- c^2 (\Delta t)^2 + (\Delta x)^2 + (\Delta y)^2 + (\Delta z)^2 < 0
\end{equation*}
we say that they are \ti{timelike-separated}, inspired by the fact that this for one happens when the two events (as measured in our present coordinate system) are at different times $\Delta t \neq 0$ but the same location $\Delta \vec{x} = 0$. In terms of signal propagation, this means that the speed required to get from one event to another by uniform motion is less than the speed of light  \begin{equation*}
\frac{\nm{\Delta \vec{x}}}{|\Delta t|} < c.
\end{equation*}
So it is certainly possible to have causal influence between two such events.

Since Lorentz transformations preserve the invariant interval, we know that inertial observers must agree as to whether two events are timelike separated. But what can they \tb{disagree} on? Two observers that differ only by a spatial rotation don't tell us much interesting about relativity, so to simplify the analysis we consider just the $t$ and $x$ coordinates, pretending we are in a (1+1)-dimensional spacetime. What might the difference $(\Delta t', \Delta x')$ look like in a boosted frame? The value of the invariant interval in the first frame \begin{equation*}
- \ell^2 := - c^2 (\Delta t)^2 + (\Delta x)^2 < 0 
\end{equation*}
must agree with that measured in the primed frame, so after moving the common minus sign around we have  \begin{equation*}
c^2 (\Delta t')^2 - (\Delta x')^2 = \ell^2
\end{equation*}
But this is exactly the equation for a hyperbola which sits inside the lightcone and asymptotes to it. The solution sets at various values of $\ell$ are represented by red lines in \ref{Fig:hyperbolas}.
\begin{figure}[ht]
\centering
\includegraphics[width=0.5\textwidth]{hyperbolas.png}
\caption{Timelike and spacelike level sets.}
\label{Fig:hyperbolas}
\end{figure}

We can follow an identical discussion for events which are \ti{spacelike-separated}, with a positive invariant interval. \begin{equation*}
- c^2 (\Delta t)^2 + (\Delta x)^2 + (\Delta y)^2 + (\Delta z)^2 > 0
\end{equation*}
These are so named because the prototypical example is two events which (in our chosen coordinate system) are simultaneous $\Delta t = 0$, differing only in their spatial location $\Delta \vec{x} \neq 0$.

Again restricting to (1+1), in any boosted frame the difference $(\Delta t', \Delta x')$ must lie on the surface
\begin{equation*}
(\Delta x')^2 - c^2 (\Delta t')^2 = \ell^2.
\end{equation*}
Examples of this surface for various values of $\ell$ are shown as blue lines in Fig. \ref{Fig:hyperbolas}.




\subsection{Causality and the Ordering of Events}
So far this all looks very symmetrical. But there is a key difference between the spacelike and timelike level sets, which becomes even clearer when we move above (1+1)-dimensional space. Just like spatial rotations about a fixed axis have a continuous parameter $\theta$ which can be gradually turned on or set to 0, Lorentz boosts have a continuous parameter $v$. So for any particle $\ell \neq 0$, no boost can make the discontinuous jump between the future-pointed part of the hyperbola \begin{equation*}
\{(\Delta t', \Delta x') \mbox{ such that } c^2 (\Delta t')^2 - (\Delta x')^2 = \ell^2 \mbox{ and } \Delta t' > 0 \}
\end{equation*}
and the past-directed part \begin{equation*}
\{(\Delta t', \Delta x') \mbox{ such that } c^2 (\Delta t')^2 - (\Delta x')^2 = \ell^2 \mbox{ and } \Delta t' < 0 \}.
\end{equation*}
Graphically this is a fairly straightforward fact: the image of $(\Delta t, \Delta x)$ under boosts is constrained to lie on a hyperbola with two disjoint sections, and these sections are individually preserved under boosts. But the physical interpretation is remarkable. Suppose we have two events with coordinates $(t_A, x_A)$ and $(t_B, x_B)$ in our original coordinate system, with difference \begin{equation*}
\LP \Delta t, \Delta x \RP = \LP t_B - t_A, x_B - x_A \RP.
\end{equation*}
So $\Delta t > 0$ means $A$ happens first in our reference frame, $\Delta t < 0$ means $B$ happens first, and $\Delta t = 0$ means our frame sees the events as simultaneous. Then our innocuous graphical result means that \tb{if two events are timelike-separated, then the order in which they occur is observer-independent.} If $c^2 (\Delta t)^2 - (\Delta x)^2 > 0$ and $\Delta t > 0$, then any primed frame will find $\Delta t ' > 0$. This is quite unlike the case for spacelike events, which could be said to be simultaneous or to happen in either order depending on our frame of reference.

In our earlier discussion of lightcones, we said an A is inside the future lightcone of B exactly when B can influence A. This is closely related to our more recent analysis, and we now have a chain of three equivalent conditions: \begin{enumerate}
\item $A$ is inside the future lightcone of $B$.
\item $A$ is timelike-separated from $B$ with $t_A > t_B$ in at least one frame.
\item $t_A > t_B$ in every inertial frame. That is, $A$ is unambiguously in the future of $B$.
\end{enumerate}
This is good news! We know that special relativity spoils some of Galilean physics's notions of absolute ordering of events, and historically it was once thought that this could lead to nonsensical predictions. But we now see that we at least avoid grandfather-style paradoxes\footnote{This is a classic media trope in which a character goes back in time to kill one of their own ancestors, thus paradoxically preventing themselves from having been born in the first place.}, since any two events which can influence each other have an unambiguous ordering.










\end{document}

