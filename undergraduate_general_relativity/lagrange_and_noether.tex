 \documentclass[12pt]{article}
\usepackage{amsmath, amssymb, mathtools, slashed, amsfonts}
\usepackage[margin=1in]{geometry}
\usepackage{tikz-cd}
\usepackage{tikz} 
\usepackage{tikz-feynman}
\tikzfeynmanset{compat=1.1.0}
\usepackage[shortlabels]{enumitem}

% Commonly used sets of numbers
\newcommand{\R}{\mathbb{R}}
\newcommand{\Z}{\mathbb{Z}}
\newcommand{\C}{\mathbb{C}}
\newcommand{\Q}{\mathbb{Q}}
\newcommand{\N}{\mathbb{N}}
\newcommand{\A}{\mathbb{A}}
\newcommand{\PR}{\mathbb{P}}
\newcommand{\F}{\mathbb{F}}


% Shortcuts for text formatting
\newcommand{\tb}[1]{\textbf{#1}}
\newcommand{\ti}[1]{\textit{#1}}

% Math fonts
\newcommand{\mf}[1]{\mathfrak{#1}}
\newcommand{\mc}[1]{\mathcal{#1}}

% Shortcuts for inner product spaces
\newcommand{\KET}[1]{\left| #1 \right\rangle }
\newcommand{\BRA}[1]{\left\langle #1 \right| }
\newcommand{\IP}[2]{\left\langle #1 \middle| #2 \right\rangle}
\newcommand{\Ip}[2]{\left\langle #1, #2 \right\rangle}
\newcommand{\nm}[1]{\left\| #1 \right\|}

% Shortcuts for the section of 3D rotations
\newcommand{\lo}{\textbf{L}_1}
\newcommand{\ltw}{\textbf{L}_2}
\newcommand{\lt}{\textbf{L}_3}


% Shortcuts for geometric vectors
\newcommand{\U}{\textbf{u}}
\newcommand{\V}{\textbf{v}}
\newcommand{\W}{\textbf{w}}
\newcommand{\B}[1]{\mathbf{#1}}
\newcommand{\BA}[1]{\hat{\mathbf{#1}}}

% Other shortcuts

\newcommand{\G}{\gamma}
\newcommand{\LA}{\mathcal{L}}

\newcommand{\LP}{\left(}
\newcommand{\RP}{\right)}

\newcommand{\PA}[2]{\frac{\partial #1}{\partial #2}}

\newcommand{\HI}{\mathcal{H}}
\newcommand{\AL}{\mathcal{A}}

\newcommand{\D}{\partial}

\newcommand{\bs}{\textbackslash}

\newcommand{\T}{\mathcal{T}}
\newcommand{\arr}{\mathcal{R}}

\numberwithin{equation}{section}
\setcounter{section}{0}





\def\Xint#1{\mathchoice
{\XXint\displaystyle\textstyle{#1}}%
{\XXint\textstyle\scriptstyle{#1}}%
{\XXint\scriptstyle\scriptscriptstyle{#1}}%
{\XXint\scriptscriptstyle\scriptscriptstyle{#1}}%
\!\int}
\def\XXint#1#2#3{{\setbox0=\hbox{$#1{#2#3}{\int}$ }
\vcenter{\hbox{$#2#3$ }}\kern-.6\wd0}}
\def\ddashint{\Xint=}
\def\dashint{\Xint-}




\begin{document}

\title{Notes on Lagrangians and Noether Charges}
\author{Mathew Calkins\\
  \textit{Center for Cosmology and Particle Physics},\\
  \textit{New York University}\\
  \texttt{mc8558@nyu.edu}.
}

\date{\today}

\maketitle

\begin{abstract}
Some supplementary discussion of Lagrangians and Noether charges.    
\end{abstract}

\tableofcontents


\section{Noether's Theorem for 3D Rotation}




\subsection{The Statement of Noether's Theorem}
In lecture we learned that, if we phrase physics in the language of actions and Lagrangians, given any continuous symmetry we can construct an associated conserved quantity. Abstractly, we might consider a system whose state at any given time is specified by a finite list of real numbers which we write as \begin{equation*}
(q_\alpha) = (q_1, \ldots, q_n)
\end{equation*}
plus the time derivatives of these quantities. For example, if we consider a system of $N$ particles moving around 3D Euclidean space and we work in Cartesian coordinates, our variables could be the $x,y,z$ components of the $N$ particles, so that $n = 3N$. We sometimes say that, by specifying every $q_\alpha$ at some fixed time, we have specified the instantaneous \ti{configuration} of our system. The space of all possible configurations for our system is called its \ti{configuration space}. Again, we note carefully that the total instantaneous state of our system (that is, the present information needed to uniquely predict future states using the equations of motion) is its location in configuration space \tb{plus} the time derivative of the same. \\
\\
\
This space can be nontrivial. For example, if our system is a simple pendulum in the $(x,y)$ plane built from a rigid rod rather than a string, then the configuration of our system at any time is specified uniquely by the angle $\theta$ measured against, say, the negative $y$. Even though the mass on our pendulum is moving through 2D Euclidean space, its configuration is specified by a single periodic real variable $\theta \sim \theta + 2 \pi$.\\
\\
\
For such a system, we can derive the equations of motion by writing down a Lagrangian which depends on the configuration variables, their time derivatives, and possibly directly time itself. \begin{equation*}
L = L(q_\alpha(t), \dot{q}_\alpha(t), t).
\end{equation*}
A \ti{symmetry} of our physics is a rule for  actively transforming all of our configuration variables $q_\alpha \to q'_\alpha$ in such a way that, if the trajectories $q_\alpha(t)$ satisfy the equations of motion, so do the new trajectories $q'_\alpha(t)$. Usually this happens because this replacement leaves the Lagrangian unchanged up to a total time derivative. \begin{equation*}
L(q'(t), \dot{q}'(t), t) = L(q(t), \dot{q}(t), t) + \frac{df}{dt}
\end{equation*}
for $f = f(q(t), \dot{q}(t), t)$\footnote{Be careful with notation here. Dots like $\dot{q}$ denote time derivatives, while primes like $q'$ denote a transformed quantity. You should pause and convince yourself that the notation $\dot{q}'$ is unambiguous, \ti{i.e.}, that the transformed value of the time derivative is the time derivative of the transformed coordinate.}. We saw in lecture that this is enough to ensure that $q_\alpha \to q_\alpha'$ takes extrema of the action to extrema of the action, and thus takes physical solutions to physical solutions.\\
\\
\
Many but not all\footnote{An example of non-continuous symmetry might be reflection across some spatial plane.} symmetries we are familiar with are \ti{continuous symmetries}. For example, consider a system of a single particle moving freely around 3D Euclidean space. The natural choice of configuration variables is just its position vector which we could write variously as \begin{equation*}
(q_\alpha) = (q_1, q_2, q_3) = (x,y,z) = \vec{x}
\end{equation*}
If the physics of this particle is unchanged under translation $x \mapsto x + a$ along the $x$ direction, then in the language above we have a symmetry \begin{align*}
q_1' & = q_1 + a, \\
q_2' & = q_2, \\
q_3' & = q_3
\end{align*}
So $q_1'$ depends not only on the original value $q_1$, but also on a continuous parameter $a$. This means we could study infinitesimal transformations by setting $a = \epsilon$ and working only to leading order in small $\epsilon$. As a slightly more complicated example we might consider rotation about the $z$ axis by an angle $\theta$. In equations, this is the transformation \begin{align*}
q_1' & = q_1 \cos \theta - q_2 \sin \theta, \\
q_2' & = q_1 \sin \theta + q_2 \cos \theta, \\
q_3' & = q_3
\end{align*}
Again, we could consider $\theta = \epsilon$ and work to leading order to study infinitesimal rotations.\\
\\
\
In both of these cases, we see that for a continuous symmetry our transformed variables $q_\alpha$ can be written as functions of all of the old values plus a continuous parameter which we write $\lambda$. That is, \begin{equation*}
q'_\alpha = G_\alpha(q, \lambda).
\end{equation*}
where we freely parametrize so that $\lambda = 0$ corresponds to no transformation at all. Note in particular that the argument $q$ on the RHS is shorthand for all of the original values. For rotations about the $z$ axis for example, $q_1'$ depends on both $q_1$ and $q_2$, not just on $q_1$. \\
\\
\
Since we have parametrized our transformation so that $\lambda = 0$ corresponds not changing anything we know that $G_\alpha(q,0) = q_\alpha$. So setting $\lambda = \epsilon$ and working out to leading order gives an expansion of the form \begin{align*}
G_\alpha(q, \epsilon) & \approx G_\alpha(q,0) + \epsilon \left. \PA{G_\alpha(q,\lambda)}{ \lambda} \right|_{\lambda = 0} \\
\ & = q_\alpha + \epsilon g_\alpha(q)
\end{align*}
where $g_\alpha$ is just shorthand for the derivative appearing in the first line. \\
\\
\
Finally we can carefully state Noether's theorem: if $q_\alpha \to q'_\alpha = G_\alpha(q,\lambda)$ is a continuous symmetry of our system, then the quantity \begin{equation*}
Q = \sum_\alpha \PA{L}{\dot{q}_\alpha} g_\alpha - f = Q(q,\dot{q}, t).
\end{equation*}
is a \ti{conserved quantity}. In other words, \begin{equation*}
\frac{d}{dt} Q(q(t), \dot{q}(t), t) = 0
\end{equation*}
whenever the trajectories $q_\alpha(t)$ obey the equations of motion. This is an important point! If someone presents you with a claimed expression for $Q$ and you attempt to show that it is constant in time, barring pathological examples you will only be able to do so if you assume the equations of motion hold.








\subsection{Angular Momentum}
This discussion was slightly abstract, so we now work out a detailed example. We consider a particle of mass $m$ moving in 3D Euclidean space subject to a potential which is invariant under rotations about arbitrary axes. This could be a body orbiting a much heavier and effectively stationary object sitting at the origin under Newtonian gravity, or something more exotic. As above, the natural choice of configuration variables is the position vector \begin{equation*}
(q_1, q_2, q_3) = (x,y,z).
\end{equation*}
Our particle's equations of motion can be derived from its Lagrangian. Since we have assumed rotational invariance the potential must depend only on the radial coordinate $r$, but otherwise could be quite general. So our Lagrangian looks like \begin{equation*}
L = \frac{1}{2} m \vec{v} \cdot \vec{v} - V(r) = \frac{1}{2} m \LP \dot{q}_1^2 + \dot{q}_2^2 + \dot{q}_3^2 \RP - V \LP \sqrt{q_1^2 + q_2^2 + q_3^2} \RP.
\end{equation*}
As before, we consider the transformation corresponding to rotation about the $z$ axis. \begin{align*}
q_1' & = G_1(q, \theta) = q_1 \cos \theta - q_2 \sin \theta, \\
q_2' & = G_2(q, \theta) = q_1 \sin \theta + q_2 \cos \theta, \\
q_3' & = G_3(q, \theta) = q_3.
\end{align*}
We want to find the conserved quantity $Q_z$ associated with this symmetry. We saw above that \begin{equation*}
Q_z = \sum_\alpha \PA{L}{\dot{q}_\alpha} g_\alpha - f = Q(q,\dot{q}, t).
\end{equation*}
so we need to find three things: the function $f$ appearing in the change of $L$, the functions $g_\alpha(q)$, and the partial derivatives of $L$ with respect to each $\dot{q}_\alpha$. \\
\\
\
First, since $r$ and $\vec{v} \cdot \vec{v}$ are unchanged under rotation, we know that our Lagrangian is preserved even without needing to allow for a total time derivative. Said otherwise, we have simply $f = 0$. \begin{equation*}
L(q'(t), \dot{q}'(t), t) = L(q(t), \dot{q}(t), t) \implies f = 0.
\end{equation*}
Next, per the prescription above we want to consider infinitesimal rotations so that we can find expressions for $g_\alpha(q)$. We recall the Taylor series expansions \begin{align*}
\cos \theta & = 1 - \frac{\theta^2}{2!} + \frac{\theta^4}{4!} - \cdots, \\
\sin \theta & = \theta - \frac{\theta^3}{3!} + \frac{\theta^5}{5!} - \cdots .
\end{align*}
From this we see that $G_1(q, \theta)$ to linear order in infinitesimal $\theta = \epsilon$ is \begin{align*}
G_1(q, \epsilon) & = q_1 \cos \epsilon - q_2 \sin \epsilon \\
\ & \approx q_1 - \epsilon q_2
\end{align*}
and similarly \begin{align*}
G_2(q, \epsilon) & = q_1 \sin \epsilon + q_2 \cos \epsilon \\
\ & \approx q_2 + \epsilon q_1.
\end{align*}
Less interestingly, \begin{equation*}
G_3(q, \epsilon) = q_3.
\end{equation*}
On the other hand, by definition we write the linearized expansions of our transformed quantities as \begin{align*}
G_\alpha(q,\epsilon) & \approx q_\alpha + \epsilon g_\alpha(q).
\end{align*}
So from our linearizations we can read off \begin{align*}
G_1(q,\epsilon) \approx q_1 - \epsilon q_2 & \implies g_1(q) = - q_2 , \\
G_2(q,\epsilon) \approx q_2 + \epsilon q_1 & \implies g_2(q) = q_1, \\
G_3(q,\epsilon) = q_3 & \implies g_3(q) = 0.
\end{align*}
Finally, from our expression for the Lagrangian \begin{equation*}
L = \frac{1}{2} m \LP \dot{q}_1^2 + \dot{q}_2^2 + \dot{q}_3^2 \RP - V \LP \sqrt{q_1^2 + q_2^2 + q_3^2} \RP
\end{equation*}
we can directly differentiate to find \begin{equation*}
\PA{L}{\dot{q}_\alpha} = m \dot{q}_\alpha.
\end{equation*}
Putting this all together, our expression for the conserved quantity associated with rotation about the $z$ axis is \begin{align*}
Q_z & = \sum_\alpha \PA{L}{\dot{q}_\alpha} g_\alpha - f \\
\ & = \PA{L}{\dot{q}_1} g_1 + \PA{L}{\dot{q}_2} g_2 + \PA{L}{\dot{q}_3} g_3 - 0 \\
\ & = \LP m \dot{q}_1 \RP \LP -q_2 \RP + \LP m \dot{q}_2 \RP \LP q_1 \RP + \LP m \dot{q}_3 \RP \LP 0 \RP \\
\ & = m (x \dot{y} - y \dot{x}) \\
\ & = x p_y - y p_x \\
\ & = \LP \vec{x} \times \vec{p} \RP_z \\
\ & = L_z.
\end{align*}
So invariance under rotation about the $z$ axis implies conservation of the $z$ component of angular momentum! This suggests several immediate generalizations, which are good to do by hand but should at least be meditated on. \\
\\
\
\tb{Generalization 1} Consider rotations about the $x$ and $y$ axes. Show that these correspond to conservation of $L_x$ and $L_y$.\\
\\
\
\tb{Generalization 2} Let $\hat{n}$ be an arbitrary unit vector and consider rotation about the corresponding axis. Show that symmetry under such rotation corresponds to the conservation of $\hat{n} \cdot \vec{L}$.\\
\\
\
\tb{Generalization 3} Consider a system of $N$ particles with positions $\vec{x}_i$ for $1 \leq i \leq N$. Suppose that for $i \neq j$ particle $i$ is interacting with particle $j$ via a potential $V_{ij}$ that depends on the distance between the two, so that our Lagrangian is \begin{equation*}
L = \sum_{i} \frac{1}{2} m_i \vec{v}_i \cdot \vec{v}_i - \sum_{i < j} V_{ij} \LP \nm{\vec{x}_i - \vec{x}_j} \RP.
\end{equation*}
Show that the physics of such a system is invariant under simultaneously rotating our entire system about the $z$ axis \begin{align*}
x_i' & = x_i \cos \theta - y_i \sin \theta, \\
y_i' & = x_i \sin \theta + y_i \cos \theta, \\
z_i' & = z_i
\end{align*}
for $\theta$ independent of $i$. What is the associated conserved quantity?




\section{Aside: Time and Space on Equal Footing}
Following some questions near the end of recitation, we also discussed in what sense Lagrangians can be made to play especially nicely with relativity. The basic tension we discussed was that one of the slogans of special relativity is ``put space and time on equal footing." Indeed, just as Newtonian physics is invariant under rotations that recombine the $x,y,z$ axes in nice ways, relativity requires additional invariance under \ti{boosts} which further mix in the $t$ axis.\\
\\
\
Naively, this seems at odds with the Lagrangian phrasing of particle mechanics. In our undergraduate mechanics courses we are taught to think of trajectories as being described by functions $\vec{x}(t)$. We take the components of the position vector $\vec{x}$ as dynamical observables with their own equations of motion, while we demote $t$ to be a parameter on which they depend, and over which we integrate to turn a Lagrangian into an action. This is reflected in the time-space asymmetry of the familiar phrasing of the Euler-Lagrange equations \begin{equation*}
\PA{L}{x_i} = \frac{d}{dt} \PA{L}{\LP \frac{dx_i}{dt} \RP}
\end{equation*}
where $x_i$ and $t$ play very different roles.\\
\\
\
To fix this problem, we sometimes say that we \ti{promote} time from a parameter to an observable. Geometrically, this means that instead of describing a trajectory with a vector-valued function $\vec{x}(t)$ moving through space \begin{equation*}
(x(t),y(t),z(t)) \in \mbox{space} \approx \R^3
\end{equation*}
we think of the path that a particle traces out in \ti{spacetime}. This means each of $x,y,z,t$ become coordinates in a four-dimensional space of \ti{events}, and over the course of a particle's history it traces out a path called its \ti{worldline}. We know that studying curves with differential calculus typically requires that we parametrize them, so in our Lagrangian framework we now work with a parametrized curve \begin{equation*}
(t(\lambda), x(\lambda), y(\lambda), z(\lambda)) \in \mbox{spacetime} \approx \R^4.
\end{equation*}
We write $\approx$ in both cases because, as we will see later, we will often need to consider spaces and spacetimes with geometries far more interesting than just flat Euclidean space.\\
\\
\
Now $\lambda$ is the independent variable for our Lagrangian description. This means our Lagrangian depends on our four spacetime components and their $\lambda$ derivatives \begin{equation*}
L = L \LP t, x, y, z, \frac{dt}{d \lambda}, \frac{dx}{d \lambda}, \frac{dy}{d \lambda}, \frac{dz}{d \lambda} \RP.
\end{equation*}
The action is supposed to be the integral of this over some trajectory, so now $L$ and $S$ are related by $\lambda$ integration \begin{equation*}
S = \int L d \lambda.
\end{equation*}
Following the same logic, we have new Euler-Lagrange equations which more clearly treat $t$ and $x_i$ on equal footing. \begin{align*}
\PA{L}{x_i} & = \frac{d}{d\lambda} \PA{L}{\LP \frac{dx_i}{d\lambda} \RP}, \\
\PA{L}{t} & = \frac{d}{d\lambda} \PA{L}{\LP \frac{dt}{d\lambda} \RP}.
\end{align*}
But this doesn't mean our work is done. In Newtonian mechanics we had to be careful to write down only Lagrangians that had rotation and translation invariance as appropriate. Now we need to figure out how to write down Lagrangians that are invariant under all the transformations of special relativity.

\end{document}
